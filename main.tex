\documentclass{article}
\usepackage[utf8]{inputenc}
\usepackage[spanish]{babel}
\usepackage{listings}
\usepackage{graphicx}
\graphicspath{ {images/} }
\usepackage{cite}

\begin{document}

\begin{titlepage}
    \begin{center}
        \vspace*{0.5cm}
            
        \Huge
        \textbf{Trabajo Calistenia}
            
        \vspace{2cm}
        \LARGE
        
            
        \vspace{1.5cm}
            
        \textbf{Diego Alejandro Osorio Jimenez}
            
        \vfill
            
        \vspace{0.8cm}
            
        \Large
        Despartamento de Ingeniería Electrónica y Telecomunicaciones\\
        Universidad de Antioquia\\
        Medellín\\
        Marzo de 2021
            
    \end{center}
\end{titlepage}

\tableofcontents
\newpage
\section{Introducción}\label{intro}
Este documento contiene el paso a paso a seguir en un proceso determinado para llevar unos objetos especificos de una posición A a una posición B, previamente concertados en clase de informatica 2

\section{Contenido} \label{contenido}
Se mostraran los paso a paso y por ultimo se tendrá una breve conclusion del trabajo 
\subsection{Paso a paso }
\textbf{1.} Identificar las dos tarjetas sobrepuestas entre si, que se encuentran en el centro de una hoja de papel apoyada sobre una superficie plana.


\textbf{2.} Las tarjetas al ser rectangulares, tienen un largo y un ancho, identificar estos y poner la mano mas habil (de ahora en adelante solo podras utilizar esa mano en el proceso a seguir) por encima de las tarjetas de manera paralela a su lado mas largo.


\textbf{3.} segun la perspectiva que se tenga en el momento de las tarjetas, identificar la parte superior central; entiendiendose superior como la altura maxima de la tarjeta y central como la media de la longitud de su anchor.


\textbf{4.} Una vez identificado este espacio de la tarjeta, tratar de poner el dedo indice por debajo de estas dos justo en la parte superior central descrita antes, con la intencion de levantar estas tarjetas, pero apoyando el debo pulgar sobre estas para evitar que deslizen entre si lo menos posible.


\textbf{5}. Si en el proceso descrito en el numeral 4 tiene alguna falla a la hora de levantar las tarjetas, como por ejemplo se deslizan totalmente entre si y quedan en posiciones diferentes; volver las tarjetas a su estado inicial y empezar nuevamente desde el numeral 1.


\textbf{6}. Una vez logres levantar las tarjetas tal como se describe en el numeral 4, apoyar estas perpendicularmente sobre el centro de la hoja (unicamente con la ayuda de la mano elegida), es importante que sea la base menor inferior la que quede apoyada.


\textbf{7. } Sin soltar las tarjetas verificar que se encuentren perfectamente sobrepuestas una sobre la otra, de tal forma que al obtener una vista frontal de estas no se logre ver la que se encuentra el la cara trasera.


\textbf{8.} luego, sin dejar caer las tarjetas, apoyar el dedo indice y pulgar en la parte superior central para obtener un buen agarre y al mismo tiempo permitir la rotación de la parte inferior de las tarjetas, entendiendo que el dedo indice debe quedar en la tarjeta de atras y el pulgar en la de adelante.


\textbf{9.} Ahora con ayuda del dedo anular vamos a tratar de separar las partes inferiores de cada una de las tarjetas, sin interrupir el agarre con los dedos indice y pulgar en la parte superior. 


\textbf{10.} Por ultimo buscamos equilibrar estas, de tal manera que se sostengan en forma de tienda indigena por si solas sobre la superficie plana, una vez logrado esto concluimos nuestro proceso.

\vspace{2cm}

\subsection{Conclusión}
Es importante destacar la importancia que tiene este trabajo a la hora de mejorar nuestra capacidad de programación, pues al tener un orden mas detallado cuando se realiza un trabajo o en nuestro caso codigo de programación, podemos lograr codigos mas eficientes y disminuir de igual manera el margen de error en estos. 



\end{document}
